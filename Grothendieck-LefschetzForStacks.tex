\documentclass{article} 
\usepackage[a4paper, total={6in, 8in}]{geometry}

\usepackage{amsthm,amsmath,amssymb,amsfonts}
\usepackage{enumerate}
\usepackage{hyperref} 
\usepackage{cleveref}
\usepackage{graphicx}
\usepackage{scalerel}

\usepackage{tikz-cd}

\newcommand{\N}{\mathbb{N}}
\newcommand{\Z}{\mathbb{Z}}
\newcommand{\Q}{\mathbb{Q}}
\newcommand{\R}{\mathbb{R}}
\newcommand{\C}{\mathbb{C}}
\newcommand{\F}{\mathbb{F}}
\newcommand{\K}{\mathbb{K}}
\newcommand{\A}{\mathcal{A}}
\newcommand{\Pro}{\mathbb{P}}

\newcommand{\calO}{\mathcal{O}}
\newcommand{\calF}{\mathcal{F}}
\newcommand{\calL}{\mathcal{L}}

\newcommand{\Spec}[1]{\text{Spec}(#1)}

\newcommand{\boldx}{\mathbf{x}}

\newcommand{\Mod}[1]{(\text{Mod } #1)}

\newcommand{\vs}[1][5]{\vspace{#1mm}}

\newcommand{\id}{\text{id}}

\newcommand{\textbfu}[1]{\textbf{\underline{#1}}}

\theoremstyle{definition}
\newtheorem{definition}{Definition}
\newtheorem{theorem}{Theorem}
\newtheorem{proposition}[theorem]{Proposition}
\newtheorem{lemma}[theorem]{Lemma}
\newtheorem{corollary}[theorem]{Corollary}
\newtheorem{remark}[theorem]{Remark}
\newtheorem{explanation}[theorem]{Explanation}


\begin{document}
    The basic outline/plan
    \begin{itemize}
        \item Lefschetz condition for ``nice'' DM stack, following Hartshorn ASAV chapter 4, begin a definition here
        \item something
    \end{itemize}    

    DM Stack basicially means that the stablizers groups are all finite.

    \section{Introduction}

    \begin{definition}
        \label{def:Lef}
        this is where the definition of \textbfu{Lef} for weighted projective stacks will go
    \end{definition}

    \begin{definition}
        \label{def:Leff}
        this is where the definition of \textbfu{Leff} for weighted projective stacks will go
    \end{definition}

    \section{Formal Geometry on Stacks}
    See this source: https://math.stanford.edu/~conrad/papers/formalgaga.pdf

    See this source: https://www.math.uchicago.edu/~emerton/pdffiles/formal-stacks.pdf

    Comment: Probably the most technical part (and the most useful) is going to be here.

    \begin{definition}
        \textbfu{Formal completion of a stack along a substack}
    \end{definition}

    \begin{proposition}
        Following 1.1 of Hartshorn ASAV, equivalence of cd(X-Y) $< n-1$ is equivalent to Lef(X,Y) and Y meets every 
        effective Cartier divisor on X
    \end{proposition}

    \begin{definition}
        \label{def:completeIntersections}
        Give the correct definition of a \textbfu{complete intersection} on a stack, probably intersection of divisors
    \end{definition}

    \begin{theorem}
        Following theorem 1.5 of Hartshorn ASAV, give the correct statement of this theorem. As a first attempt take X to be
        equal to the weighted projective stack. Something about fake weighted projective stacks (maybe?)

        A complete intersection, Y, has Leff(X,Y) assuming that that Y is dimension at least 2.
    \end{theorem}

    \section{Application to the Picard Group}
    
    \begin{theorem}
        Following theorem 3.1 in Hartshorn ASAV
        \begin{proof}
            What we need for the proof is:
            \begin{itemize}
                \item Deformation theory argument
                \item third point of Hartshorn's version deals with $H^i(Y,I^n/I^{n+1})$ this is where the deformation theory comes in
                \item We have a line bundle $L$ on $Y$, we want to lift it to $X$
                \item There are a series of ``natural'' maps $Pic(X) \to Pic(U) \to Pic(\widehat{X}) \to Pic(Y)$. Need to show that
                      they are all isomorphisms and that their composition is the ``natural'' map $Pix(X) \to Pic(Y)$ that comes
                      from the pull back $\calL \mapsto i^*(\calL)$ where $i: Y \hookrightarrow X$ is the inclusion of $Y$ into $X$.
            \end{itemize}
        \end{proof}
    \end{theorem}

    \section{Cohomological properties of DM Stacks}
    Serre duality for tame DM stacks. (Tame basically means that if characteristic = p, then stablizer groups do not have order 
    $p^n$ for any $n$)

    See: https://link.springer.com/article/10.1007/s40687-022-00367-7

    Serre's theorem on global generation on Stacks

    See: https://arxiv.org/abs/1306.5418

    \begin{theorem}
        Generalizing theorem 5.18 in Hartshorn's AG
    \end{theorem}

    \section{Calculate the cohomology of line bundles on weighted projective stack}
    \begin{theorem}
        Generalizing theorem 5.1 in Hartshorn's AG
    \end{theorem}    

\end{document}